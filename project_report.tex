\documentclass[conference]{IEEEtran}
\usepackage{cite}
\usepackage[pdftex]{graphicx}
\usepackage[cmex10]{amsmath}
\usepackage{algorithmic}
\usepackage{array}
\usepackage{mdwmath}
\usepackage{mdwtab}
\usepackage{eqparbox}
\usepackage[caption=false,font=footnotesize]{subfig}
\usepackage{fixltx2e}
\usepackage{stfloats}
\usepackage{url}
\usepackage{listings}
\usepackage{todonotes}

\hyphenation{}


\begin{document}
\title{Softwareprojekt Telematik\\POSIX Compliant Shell\\Final Report}


\author{\IEEEauthorblockN{Robert Fels}
\IEEEauthorblockA{robert.fels@fu-berlin.de}
\and
\IEEEauthorblockN{Martine Lenders}
\IEEEauthorblockA{mlenders@inf.fu-berlin.de}
\and
\IEEEauthorblockN{Jakob Pfender}
\IEEEauthorblockA{jakob.pfender@fu-berlin.de}}

\maketitle


\begin{abstract}
\end{abstract}

\section{Introduction}
\label{sec:Introduction}

\subsection{Goals}
\label{sub:Goals}
The goal of this project was to port a POSIX compliant shell, such as
dash\cite{dash} or BusyBox\cite{busybox}, to RIOT. In order to do so,
RIOT's POSIX compliance initially needed to be improved.

\subsection{Motivation}
\label{sub:Motivation}
The motivation for this project is twofold: On the one hand, the direct
advantage of having a lightweight shell on RIOT is that it provides an
execution environment for shell scripts. This means that developers are
able to write small tasks as shell scripts as opposed to writing them in
C, compiling and flashing them onto the device. In addition to this,
existing shell scripts can be easily ported to RIOT. The other aspect
that motivates this project is the fact that dash is POSIX compliant,
which implies that systems it runs on are also expected to be POSIX
compliant to some degree. While the POSIXization of RIOT is already
underway, it is not fully compliant yet. Porting a POSIX compliant shell
to RIOT would go a long way toward improving this compliance.

\subsection{Context}
\label{sub:Context}
When talking about shells, the dominant aspect is more often than not
their interactivity, i.e. providing a command line for the user to
interact directly with the system, executing commands, producing output
et cetera. However, this is not the focus of this project. There is no
real need for an interactive shell in the embedded context, where
devices are ideally programmed once and then left to perform their tasks
indefinitely, producing non-interactive output. For those cases where
interactivity is desired, RIOT already provides a minimal interactive
shell, which is not much more than a command line interface and can't
execute shell scripts. The focus here, as described in section
\ref{sub:Motivation}, is primarily to provide an execution environment
for shell scripts as a lightweight alternative to compiling and flashing
full-fledged C applications.

\section{Approach}
\label{sec:Approach}
The first decision to be made was which shell to implement. We decided to
implement dash because BusyBox is itself a derivative of dash, but
brings with it a slew of features (such as an SSH server) that we deemed
superfluous to our needs.

We divided the work to be done in order to port dash to RIOT into two
rough phases. The first phase is the compilation, i.e. getting dash to
build on RIOT. The second phase is the actual functionality needed to
run dash and execute shell scripts.

\subsection{Compilation}
\label{sub:Compilation}
Getting dash to compile on RIOT means supplying all of the POSIX-API
functions required by dash, as specified in its includes. We further
divided this phase up into the identification of these missing headers
and their respective implementation.

Some of the needed headers are already supplied by the newlib and/or the
msp430-libc. However, a number of header files were not yet implemented
in either of these libraries, necessitating their reimplementation for
RIOT. The added benefit of implementing these headers is not only
enabling dash to build on RIOT, but also furthering RIOT's POSIX
compliance, which in turn makes future porting of other POSIX-compliant
software to RIOT easier.

Identifying the headers that needed to be implemented was very helpful
in compartmentalizing the initial work into several small tasks with
minimal overlap. The following is an overview of headers we identified
as needing work:

\begin{itemize}
    \item dirent.h\footnotemark[1]
    \item errno.h\footnotemark[1]
    \item fcntl.h\footnotemark[1]
    \item fnmatch.h\footnotemark[1]
    \item glob.h\footnotemark[1]
    \item paths.h\footnotemark[1]
    \item pwd.h\footnotemark[1]
    \item sys/param.h\footnotemark[1]
    \item sys/resource.h\footnotemark[1]
    \item sys/stat.h\footnotemark[1]
    \item sys/time.h\footnotemark[1]
    \item sys/times.h\footnotemark[1]
    \item sys/wait.h\footnotemark[1]
    \item termios.h\footnotemark[1]
    \item unistd.h\footnotemark[1]
    \item setjmp.h\footnotemark[2]
    \item signal.h\footnotemark[2]
    \item stdio.h\footnotemark[3]
    \item sys/ioctl.h\footnotemark[4]
    \item stdarg.h
    \item stddef.h
\end{itemize}

\footnotetext[1]{available in newlib, not in msp430-libc}
\footnotetext[2]{available but some functions need implementation}
\footnotetext[3]{available but some functions not available in msp430-libc}
\footnotetext[4]{not POSIX compliant}

As you can see, most of these headers are available in newlib, but not
in msp430-libc. Therefore, we decided to primarily focus on support for
non-MSP430 first and add the compliance for MSP430 later.

\subsection{Functionality}
\label{sub:Functionality}
After providing the prerequisites for building dash on RIOT and thus
also providing the level of POSIX compliance expected by dash, there are
still several steps needed to be taken in order to actually have
a usable execution environment for shell scripts. There are several
things a shell such as dash expects of the system it runs on. These
include process handling and signaling, the ability to call
\texttt{fork()} and \texttt{kill()}, some sort of file system, as
well as wrappers for POSIX functions. We will now describe these and our
approach to implementing them in detail.

\subsubsection{Process handling and signaling}
\label{ssub:Process handling and signaling}
\todo{do we talk about IPC here?}

\subsubsection{\texttt{fork()}}
\label{ssub:fork}
Implementing \texttt{fork()} turned out to be not exactly trivial, since
RIOT is currently missing the capability to directly read and/or set the
instruction pointer, which is integral to \texttt{fork()}. We
implemented the needed functions for the Cortex-M3 processor; the other
CPUs are still missing this functionality.

In order to allow forking processes, we created a pool of
pre-initialized memory that can be used to copy stacks of forked
processes. Forking a thread then allocates the next free block in this
pool, duplicates the current thread's stack and stack pointer and sets
the newly created thread's instruction pointer to a point after the
original thread's return.

\subsubsection{\texttt{kill()}}
\label{ssub:kill}
Shells expect to be able to kill processes ``from the outside``, i.e.
the scheduler has the ability to terminate threads other than the one
that is currently running. Implementing this functionality was a simple
matter of supplying a \texttt{sched\_task\_kill()} function to kill
processes using their PID and refactoring the old
\texttt{sched\_task\_exit()} (used to terminate the currently active
process) to call this new function.

\subsubsection{File system}
\label{ssub:File system}
In UNIX philosophy, everything is a file. This is also apparent in POSIX
and in the implementation of POSIX compliant shells such as dash.
However, RIOT currently does not provide a file system. This was
a problem we needed to adress in order to provide a function shell
script environment.

We designed a file system interface which was loosely based on
FUSE\cite{fuse}. We informed ourselves about possible file systems to
implement in order to present shell scripts to RIOT. These were the
candidates we considered:

\begin{itemize}
  \item \emph{FAT}\cite{fat} -- problematic because of licensing issues. The FAT
    specification is owned by Microsoft, which does not allow free use
    of these specifications and ``associated intellectual
    property``\cite{fatlicense}.
  \item \emph{UFFS}\cite{uffs}, \emph{nuttX}\cite{nuttx} -- these are
    not used very often. They are quite new and are still missing
    thorough testing.
  \item \emph{CFS}\cite{cfs} -- seems to be the best available choice in
    our opinion. It is a minimal, yet fully functional, POSIX compliant
    file system. Since it is part of the Contiki project\cite{contiki},
    it is primarily aimed at sensor devices equipped with flash memory
    or EEPROM. Although it is part of Contiki, there are no
    Contiki-specific abstractions used in its implementation, which
    means that porting it to RIOT is feasible.
\end{itemize}

\subsubsection{Wrappers for POSIX functions}
\label{ssub:Wrappers for POSIX functions}
\todo{this isn't finished, right? what is the status?}

\subsection{Issues}
\label{sub:Issues}
The primary issue impeding our work on the actual dash implementation
was the fact that close to none of the POSIX prerequisites expected by
dash were supplied to begin with. This meant that most of our initial
work was only tangential to dash, being concerned with bringing RIOT up
to a level of POSIX compliance sufficient for dash to work. In addition
to the hard POSIX requirements, several other features expected by dash
were also missing from RIOT; most importantly, RIOT's process handling
didn't support signaling or forking and there was no file system. These
were obstacles in our path to a working dash implementation that we
needed to overcome. Some of these later turned out to be bigger
impediments than initially expected, with the effect that our planned
schedule couldn't be kept.

\section{Achievements}
\label{sec:Achievements}
During our work on the project, we were able to define all of the POSIX
prototypes needed for integration of dash into RIOT. We implemented
capabilities for the RIOT scheduler to kill processes other than the
currently running one; the corresponding pull request, however, is
currently stalled because there is a major restructuring of RIOT's IPC
in the works, which takes priority over the scheduler restructuring. We
also laid the groundwork for providing a \texttt{fork()} system call,
implementing the high-level call. As a result of defining the POSIX
prototypes, we are now able to compile dash on RIOT. We have provided
a working file system interface based on FUSE and have started
investigations in how to best implement a file system for RIOT. We also
provide SCFS, the \texttt{shell\_commands} file system, giving access to
already implemented shell handlers in RIOT.

\subsection{Comparison to original plan}
\label{sub:Comparison to original plan}
Our original mission statement was to provide all of the missing POSIX
functionality required by dash to compile and run on RIOT. This goal was
only partially achieved -- specifically, we did not implement wrappers
for the POSIX functionality and there is no functioning file system yet.
We underestimated these requirements in our original assessment.

\subsection{Pull Requests}
\label{sub:Pull Requests}
We submitted the following pull requests as part of this project:

\begin{itemize}
  \item \textbf{\#1166 Extend existing POSIX headers by using
    \texttt{-include} flag}: Newlib and msp430-clib have some POSIX
    headers, but do not define all of the POSIX prototypes, macros,
    types et cetera.  Moreover, they do not supply any means to include
    them. This pull request attempts to solve the problem by including
    them globally into every compiling unit using the \texttt{-include}
    flag.
  \item \textbf{\#1243 shell: Return-based error-handling for shell
    handlers}: Shell handlers can be interpeted as shell commands, with
    the possibility to interpret the return value. Also has the added
    benefit of enabling testing of shell handlers via unit tests.
  \item \textbf{\#1245 core: Allow killing tasks using PID}: Allows the
    scheduler to kill tasks other than the currently active task by
    using the target task's PID. This pull request is currently stalled
    because RIOT's IPC is being reworked.
  \item \textbf{\#1265 fs: Filesystem module (WIP)}: Creates a common
    interface to be used by all filesystems.
  \item \textbf{\#1384 Implement Fork}: Implements \texttt{fork()} by
    using a dedicated stack pool for forked processes.
  \item \textbf{\#1418 sys: scfs: Add filesystem that allows to mount
    shell\_command array as directory of executable files}: Depends on
    \textbf{\#1265}.
\end{itemize}

\section{Perspectives}
\label{sec:Perspectives}
We still need to work on the comments and suggestions made by the RIOT
team regarding our pull requests. This includes better integration with
the rest of the codebase, refactoring, documentation and, most
importantly, tests. We also plan to finish the missing parts of what we
originally intended to implement during the course of the project,
namely the file system and the missing POSIX functionalities.

\section{Conclusion}
\label{sec:Conclusion}
Our time management was not optimal. We spent more time than expected on
providing POSIX funtionality and did not anticipate the workload
associated with the missing features, especially the filesystem. All in
all, we believe we laid good groundwork for furthering RIOT's POSIX
compliance and providing an execution environment for shell scripts,
even though not all of our goals were met within the planned timeframe.

\section*{Acknowledgment}
\label{sec:Acknowledgement}
We would like to thank Emmanuel Baccelli, Oleg Hahm and Hauke Petersen
for initiating and managing the project and always being available for
feedback and tips. We would also like to thank Stefan Pfeiffer for
consulting with us regarding the file system and Ren\'{e} Kijewski,
Christian Mehlis and Ludwig Ortmann for providing feedback on our pull
requests.

\bibliographystyle{IEEEtran}
\bibliography{IEEEabrv,bibliography}

\end{document}
