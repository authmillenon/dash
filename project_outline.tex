\documentclass[11pt,
  a4paper,
  BCOR=7mm
]{scrartcl}
\usepackage[utf8]{inputenc}
\usepackage[english]{babel}
\usepackage{tocbasic}
\usepackage[headsepline]{scrpage2}
\usepackage{fancyref}
\usepackage{xcolor}
\usepackage[hyphens]{url}
\usepackage{listings}
\usepackage[pdftex]{graphicx}
\usepackage{courier}
\usepackage{amsmath}
\usepackage{dsfont}
\usepackage[colorlinks=false, linktocpage]{hyperref}
\setlength\parindent{0pt} 

\usepackage{DejaVuSansMono}
\lstset{language=C}
\lstset{basicstyle=\ttfamily}
\lstset{breaklines=true}
\lstset{keywordstyle=\color{purple}}

\usepackage{geometry}
\usepackage{tikz}
\usetikzlibrary{calc,arrows}

\newdimen\XCoord
\newdimen\YCoord

\newcommand*{\ExtractCoordinate}[1]{\path (#1); \pgfgetlastxy{\XCoord}{\YCoord};}%

\author{Robert Fels, Martine Lenders, Jakob Pfender}
\title{Softwareprojekt Telematik:\\POSIX compliant shell\\Project Outline}

\begin{document}

\maketitle

% \newpage

\section*{Overview and Motivation}
\label{sec:overview_motivation}
Our task is to port a lightweight POSIX compliant shell, preferably
dash, to RIOT.

\subsection*{What is RIOT?}
\label{sub:what_is_riot}
RIOT is an operating system for the Internet of
Things\footnote{\url{https://en.wikipedia.org/wiki/Internet_of_Things}}.
Unlike other embedded operating systems such as Contiki or TinyOS, RIOT
programs can be written in pure C or C++, enabling cross-platform
compatibility and easy portation of existing code. It is already quite
feature-rich, with working implementations of 6LoWPAN, RPL, and many
more. However, it is not yet fully POSIX compliant (see below).
Furthermore, there exists only a minimal system shell and no file
system. These are all points we aim to address within this software
project.

\subsection*{What is POSIX compliance?}
\label{sub:what_is_posix}
POSIX stands for ``\textbf{P}ortable \textbf{O}perating \textbf{S}ystems
\textbf{I}nterface`` and describes a family of standards specified by
the IEEE for maintaining compatibility between operating systems. It
defines the API, shells and utility interfaces for software
compatibility with variants of Unix and other operating systems. The
standards for the user command line and scripting interface, which form
the most important part of POSIX for this software project, were
originally based on the Korn shell. However, many other system shells,
such as dash (see below), are also compliant with the shell standards
set by POSIX.

\subsection*{What is dash?}
\label{sub:what_is_dash}
Dash is the Debian Almquist shell, a version of the Almquist shell,
which was originally written by Kenneth Almquist in the 1980s. Dash is
notable for its fast shell script execution; Debian and its derivatives
use it as the default /bin/sh for shell scripts, even though bash is
used as the default interactive shell. Since dash is also very
lightweight, it is very popular in the embedded sector. Our goal is to
port dash to RIOT in order to provide a lightweight, fast execution
environment for shell scripts.


\section*{Problems to solve}
\label{sec:problems}
In order to port dash to RIOT, we obviously need to bring it to compile
on RIOT. This will be the first and most important goal of our software
project. To achieve it, we need to make sure all the prerequisites for
building dash are in place. This basically means supplying all the
POSIX-API functions required by dash, as specified in the includes. Some
of these are already supplied by the newlib and/or the msp430-libc;
however, a number of header files are not yet implemented in either of
these libraries. This means that we will have to reimplement them for
RIOT. This will obviously have the added benefit of not only allowing us
to compile dash for RIOT, but also furthering RIOT's POSIX compliance,
which will make future porting of other POSIX-compliant software easier.
Our first task -- which we have already started -- is thus to identify
the headers and functions currently missing from the RIOT libraries and
implementing them. This will provide us with a nice set of more-or-less
compartmentalized tasks that can be easily divided among team members.

The next problem will be implementing the actual functionality needed for
a shell to work. Getting dash to compile is one thing; however, this
doesn't mean that users will be able to execute shell scripts right
away. In order to use dash, other prerequisites must be met. First and
most important among these are a file system and environment variables.
After we have successfully compiled dash for RIOT, the next task will be
identifying all missing functionality and implementing it.

\section*{Tasks}
\label{sec:tasks}
The whole project can be divided into two main phases -- compilation and
functionality -- with their respective subphases:

\begin{itemize}
  \item \textbf{Phase 1}: Compilation
    \begin{itemize}
      \item \textbf{Subphase 1}: Identify missing headers
      \item \textbf{Subphase 2}: Implement missing headers
    \end{itemize}
  \item \textbf{Phase 2}: Functionality
    \begin{itemize}
      \item \textbf{Subphase 3}: Identify missing functionality
      \item \textbf{Subphase 4}: Implement missing functionality
      \item \textbf{Subphase 5}: Test
    \end{itemize}
\end{itemize}

\subsection*{First main phase: Compilation}
\label{sub:compilation}
The expected outcome of this phase is a successful compilation of dash
on RIOT.

\subsubsection*{Subphase: Identification of missing headers}
\label{ssub:headers_identification}
This task is already done. We have compiled a list of headers that
neither newlib nor msp430-libc provide:

\begin{itemize}
    \item dirent.h\footnotemark[1]
    \item errno.h\footnotemark[1]
    \item fcntl.h\footnotemark[1]
    \item fnmatch.h\footnotemark[1]
    \item glob.h\footnotemark[1]
    \item paths.h\footnotemark[1]
    \item pwd.h\footnotemark[1]
    \item setjmp.h\footnotemark[2]
    \item signal.h\footnotemark[2]
    \item stdarg.h
    \item stddef.h
    \item stdio.h\footnotemark[3]
    \item sys/ioctl.h\footnotemark[4]
    \item sys/param.h\footnotemark[1]
    \item sys/resource.h\footnotemark[1]
    \item sys/stat.h\footnotemark[1]
    \item sys/time.h\footnotemark[1]
    \item sys/times.h\footnotemark[1]
    \item sys/wait.h\footnotemark[1]
    \item termios.h\footnotemark[1]
    \item unistd.h\footnotemark[1]
\end{itemize}

\footnotetext[1]{available in newlib, not in msp430-libc}
\footnotetext[2]{available but some functions need implementation}
\footnotetext[3]{available but some functions not available in msp430-libc}
\footnotetext[4]{not POSIX compliant}

Most of these headers are available in newlib, but not in msp430-libc.
This is why we will focus on support for non-MSP430 first and add the compliance
for MSP430 later, if there is time.

\subsubsection*{Subphase: Implementation of missing headers}
\label{ssub:headers_implementation}
We need to implement the missing headers and the functionality they
provide. After identifying possible dependencies between these headers,
we will distribute the implementation tasks among ourselves.

% \pagebreak

A rough preliminary distribution of tasks based on features can be seen below. 
The headers in \emph{italics} are the ones needed for msp430.

\begin{itemize}
    \item   \textbf{Jakob:} (Pattern-matching)
            \begin{itemize}
                \item   stdarg.h
                \item   \emph{fnmatch.h}
                \item   \emph{glob.h}
            \end{itemize}
    \item   \textbf{Martine:} (Filesystem)
            \begin{itemize}
                \item   File descriptor definitions
                \item   \emph{fcntl.h}
                \item   \emph{dirent.h}
                \item   \emph{paths.h}
                \item   \emph{pwd.h}
            \end{itemize}
    \item   \textbf{Robert:} (Signaling + remainder)
            \begin{itemize}
                \item   stddef.h (these are only a few constant macros)
                \item   signal.h
                \item   setjmp.h
                \item   \emph{errno.h (+ getting perror somehow in stdio.h)}
            \end{itemize}
\end{itemize}

\subsection*{Second main phase: Functionality}
\label{sub:functionality}
The expected outcome of this phase is a working implementation of dash
running on RIOT that provides all the funtionality expected of dash.

\subsubsection*{Subphase: Identification of missing functionality}
\label{ssub:functionality_identification}
We will identify all functionality expected by dash, such as a file
system and environment variables.

\subsubsection*{Subphase: Implementation of missing functionality}
\label{ssub:functionality_implementation}
We need to implement all missing functionality. After identifying
possible dependencies, we will distribute the implementation tasks among
ourselves.

\subsubsection*{Subphase: Testing}
\label{ssub:testing}
In order to test our port, we will design a series of tests in order to
verify that all functionality expected by dash is provided.


\section*{Timeline}
\label{sec:timeline}
Our proposed project timeline looks like this:
\newline

\pgfmathsetmacro{\mintime}{0}
\pgfmathsetmacro{\maxtime}{11}
\newcommand{\timeunit}{Weeks}
\pgfmathtruncatemacro{\timeintervals}{11}
\pgfmathsetmacro{\scaleitemseparation}{4}
\pgfmathsetmacro{\timenodewidth}{2cm}
\newcounter{itemnumber}
\setcounter{itemnumber}{0}
\newcommand{\lastnode}{n-0}

\newcommand{\timeentry}[2]{% time, description
\stepcounter{itemnumber}
\node[below right,minimum width=\timenodewidth] (n-\theitemnumber) at (\lastnode.south west) {#2};
\node[right] at (n-\theitemnumber.east) {};

\edef\lastnode{n-\theitemnumber}

\expandafter\edef\csname nodetime\theitemnumber \endcsname{#1}
}

\newcommand{\drawtimeline}{%
    \draw[very thick,-latex] (0,0) -- ($(\lastnode.south west)-(\scaleitemseparation,0)+(0,-1)$);
    \ExtractCoordinate{n-\theitemnumber.south}
    \pgfmathsetmacro{\yposition}{\YCoord/28.452755}
    \foreach \x in {1,...,\theitemnumber}
    {   \pgfmathsetmacro{\timeposition}{\yposition/(\maxtime-\mintime)*\csname nodetime\x \endcsname}
        \draw (0,\timeposition) -- (0.5,\timeposition) -- ($(n-\x.west)-(0.5,0)$) -- (n-\x.west);
    }
    \foreach \x in {0,...,\timeintervals}
    {   \pgfmathsetmacro{\labelposition}{\yposition/(\maxtime-\mintime)*\x}
        \node[left] (label-\x) at (-0.2,\labelposition) {\x\ \timeunit};
        \draw (label-\x.east) -- ++ (0.2,0);
    }   
}

\begin{tikzpicture}
\node[inner sep=0] (n-0) at (\scaleitemseparation,0) {};
\timeentry{2.5}{May 11th: Successful compilation}
\timeentry{3.5}{May 18th: Identify missing functionality}
\timeentry{8}{June 29th: Implement functionality}
\timeentry{10}{July 8th: Finish testing}
\timeentry{11}{July 15th: Finish project}
\drawtimeline
\end{tikzpicture}

\end{document}

\end{document}
